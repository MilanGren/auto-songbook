\beginsong{Cukrářská bossanova}[by={Jaromír Nohavica}]

\beginverse
Můj \[Cmaj7]přítel \[C#dim]snídá sedm \[Dmi7]kremrolí\[G7]
a když je \[Cmaj7]spořádá, dá si \[C#dim]repete,
\[Dmi7]cukrlát\[G7]ko,
on totiž \[Cmaj7]říká: \[C#dim]Dobré lidi zuby \[Dmi7]nebolí\[G7]
a je to \[Cmaj7]paráda, chodit \[C#dim]po světě
a \[Dmi7]mít, \[G7]mít v ústech \[Cmaj7]sladko. \[C#dim, Dmi7, G7]
\endverse
\beginchorus
\[Cmaj7]Sláva, \[C#dim]cukr a \[Dmi7]káva a půl litru \[G7]becherovky,
\[Cmaj7]hurá, hurá, \[C#dim]hurá, půjč mi \[Dmi7]bůra, útrata dnes \[G7]dělá čtyři stovky,
všechny \[Cmaj7]cukrářky z celé \[C#dim]republiky
na něho \[Dmi7]dělají slaďounké \[G7]cukrbliky
a on jim \[Emi7]za odměnu zpívá \[A7]zas a znovu
\[Dmi7]tuhletu \[G7]cukrářskou bossa-no\[Cmaj7]vu. \[C#dim, Dmi7, G7]
\endchorus
\beginverse
Můj přítel Karel pije šťávu z bezinek,
říká, že nad ni není,
že je famózní, glukózní, monstrózní, ať si taky dám,
koukej, jak mu roste oblost budoucích maminek
a já mám podezření,
že se zakulatí jako míč
a až ho někdo kopne, odkutálí se mi pryč
a já zůstanu sám, úplně sám.
\endverse
\beginchorus
Sláva ...
\endchorus
\beginverse
Můj přítel Karel Plíhal už na špičky si nevidí,
postava fortelná se mu zvětšuje,
výměra tři ary,
on ale říká: glycidy jsou pro lidi,
je prý v něm kotelna, ta cukry spaluje,
někdo se zkáruje, někdo se zfetuje
a on jí bonpari, bon, bon, bon, bonpari.
\endverse
\beginchorus
Sláva ...
\endchorus

\endsong
